\documentclass[a4paper]{article}
\usepackage[margin=1in]{geometry}

\usepackage{indentfirst}
\usepackage{graphicx}
\usepackage{hyperref}
\usepackage{float}
\usepackage{amsmath}
\usepackage{listings}
\usepackage{tabularx}
\usepackage{makecell}
\usepackage[usestackEOL]{stackengine}
\usepackage{changepage}
\usepackage{placeins}
\usepackage{siunitx}
\usepackage{xcolor}
\usepackage{caption}
\usepackage{subcaption}
\usepackage[english]{babel}
\usepackage[utf8]{inputenc}
\usepackage[document]{ragged2e}

\captionsetup{labelformat=empty,justification=centering}
\edef\tmp{\the\baselineskip}
\setstackgap{L}{\tmp}
\sisetup{output-exponent-marker=\ensuremath{\mathrm{e}}}
\definecolor{numbers_colour}{rgb}{0.5,0.5,0.5}
\definecolor{mygreen}{rgb}{0,0.6,0}
\definecolor{background_colour}{rgb}{0.95,0.95,0.92}
\lstset{basicstyle=\footnotesize, backgroundcolor=\color{background_colour}, keywords={,begin, while, do, end, if, then, else, for, return, true, break, false}, numbers=left, numberstyle=\tiny\color{numbers_colour}, commentstyle=\color{mygreen}, morecomment=[n]{/*}{*/}, keywordstyle=\color{magenta}, captionpos=b, emph=[1]{currentGen, number_of_gens, pop, best_sol, possible_sol, counter, optimal_tour, opt_poz, currentInstance, candidate, t, bestValue, i ,numberOfCities, otherPos, currentValue, foundNewCandidate},emphstyle=[1]{\itshape}, emph=[2]{HillClimbingBI},emphstyle=[2]{\sffamily}, mathescape=true}

\title{T3 - A Study on the Performance of Both Genetic Algorithms and Heuristics in Solving the Traveling Salesman Problem}
\author{Daniș Ciprian\\
	Oloieri Alexandru \\
	Second Year, A2}
\date{December 6, 2020}

\begin{document}
\maketitle
\justify
\section{Introduction}

The \textbf{travelling salesman problem}\cite{tsp} (also called the \textbf{travelling salesperson problem} or \textbf{TSP}) asks the following question: "Given a list of cities and the distances between each pair of cities, what is the shortest possible route that visits each city and returns to the origin city?" It is an \textbf{NP-hard} problem in combinatorial optimization, important in operations research and theoretical computer science.

In this study we aim to analyze the performance of both a Genetic Algorithm and a heuristic in solving several instances of TSP, compare the results of the two methods and draw an inference as to which algorithm yields optimal values for a certain instance.

\section{Method}

\subsection{Simulated Annealing}

We will use a version of the algorithm in which, at each step, we choose as the next candidate the first neighbor that meets the search criteria (Simulated Annealing First Improvement).

\begin{lstlisting}[frame=single][caption={Simulated Annealing Pseudocode}]
begin
	candidate := generate_candidate() 
	/*generate a random permutation of the numbers 0,1,...,numberOfCities-1*/
	t := 1000
	bestValue = Euclidian_2D(currentInstance, candidate)
	while (true) do begin
		foundNewCandidate := false
		for (i:=0; i<numberOfCities-1; ++i) do
			for (j:=i+1; j<numberOfCities;++j) do
				swap(candidate[i], candidate[j])
				/*"candidate" stores a neighbor of the current candidate*/
				currentValue := Euclidian_2D(currentInstance, candidate)
				if currentValue < bestValue 
				|| random(0,1) < exp(-abs(currentValue-bestValue)/t) then
					bestValue := currentValue
					foundNewCandidate := true
					break
				swap(candidate[i], candidate[j])
			if foundNewCandidate then
				break
		t := 0.9*t;
		if foundNewCandidate = false then
			break
	return bestValue
end
\end{lstlisting}
\justify
\textbf{Legend:}\\
{\itshape candidate} - current candidate (a permutation of numbers 0,1,...,numberOfCities)\\
{\itshape Euclidian\_2D}() - function used to determine the "weight" of a candidate for a TSP instance\\
{\itshape bestValue} - the best weight of a permutation found

\subsection{Genetic Algorithm}

We will work in the following experiment with an improved Genetic Algorithm that makes use of hypermutation and another heuristic (in our case, Hill Climbing Best Improvement).

\begin{lstlisting}[frame=single][caption={Genetic Algorithm Pseudocode}]
begin
	init_pop() /*initialize the population*/
	set_mold()/*create a list of all the "cities"*/ 
	currentGen := 1
	optimal_tour := decodeElem(pop[det_optimal_chromosome()])
	best_sol := Euclidian_2D(currentInstance,optimal_tour)
	counter := 1
	while (currentGen <= number_of_gens) do begin
		hypermutation(counter)
		crossover()
		selection()
		opt_poz := det_optimal_chromosome()
		possible_sol := Euclidian_2D(currentInstance,decodeElem(pop[opt_poz]))
		currentGen++
		if possible_sol < best_sol then
			best_sol := possible_sol
			if Euclidian_2D(currentInstance,optimal_tour) > best_sol then
				optimal_tour := decodeElem(pop[opt_poz])
			counter := 1
		else counter++
	HillClimbingBI(optimal_tour)
end
\end{lstlisting}
\justify
\textbf{Legend:}\\
{\itshape pop} - current population/array of solution\\
{\itshape decodeElem}() - function used to obtain the permutation of the "cities" from a chromosome; see the \textbf{Experiment} section for more\\
{\itshape Euclidian\_2D}() - function used to determine the "weight" of a decoded chromosome; see the \textbf{Experiment} section for more\\
{\itshape best\_sol} - the current best value of \emph{Euclidian\_2D}() in all the population, throughout the generations\\
{\itshape possible\_sol} - a candidate for the title of {\itshape best\_sol} in a generation\\
{\itshape optimal\_tour} - the best value of the best chromosome throughout all generations\\
{\itshape currentInstance} - the instance of TSP with which we are currently working\\
{\itshape opt\_poz} - the position of the best chromosome in the population in a generation\\
{\itshape det\_optimal\_chromosome}() - function used to determine the position of the best chromosome in a population in a generation\\
{\itshape counter} - the number of generations throughout which the value of {\itshape best\_sol} has not changed\\
{\sffamily HillClimbingBI}({\itshape optimal\_tour}) - implementation of Hill Climbing Best Improvement\cite{heuristic} in which the starting point, which was generated randomly, is replaced by {\itshape optimal\_tour}\\
{\itshape selection}(), {\itshape hypermutation}()\cite{mutation} and {\itshape crossover}() - see the \textbf{Experiment} section

\section{Experiment}

\subsection{General}

We will work with 4 instances of symmetric TSP (precisely with \textit{eil51}\cite{instances}, \textit{ts225}\cite{instances}, \textit{krod100}\cite{instances} and \textit{st70}\cite{instances}), in which the distance from node i to node j is the same as from node j to node i. Moreover, the instances have the edges' weight represented as Euclidian 2D distances. Therefore, in {\itshape Euclidian\_2D}() we will compute the weight of a circuit of all the "cities".

\subsection{Simulated Annealing}

The candidates will be represented as arrays of integers, which represent permutations of the order in which the "cities" will be visited ($n$ cities). A neighbor of the current candidate will be a permutation of size $n$ in which $n-2$ elements have the same position, and $2$ elements are swapped.

For each map, we will run the algorithm 30 times, and for each run we will have 100 iterations of the Simulated Annealing metaheuristic (so we will get $3000$ results).

\subsection{Genetic Algorithm}

Each chromosome in the population will be represented as a list of positions that the nodes can have in a permutation. More exactly, each element in the chromosome will be an integer in the interval $[0,n-1-i]$, where $n$ represents the number of nodes and $i$ the position of the element in the chromosome.

In {\itshape hypermutation}(), {\itshape counter} keeps count of how many generations have gone by since the value of {\itshape best\_sol} has been changed. If the value reaches the threshold then, instead of performing a generic mutation (a mutation with a small rate, e.g. $0.015$), we will mutate the population with a rate of $0.5$. This way, we "reset" the population so that the processing of the chromosomes does not stagnate around an accumulation point.

In {\itshape crossover}(), because of the structure of the chromosomes, which allows for repetitive values, we will perform a generic lossless cut-point crossover (both the "children" and their "parents" will be included in the population, resulting in an "extended" population).

In {\itshape selection}(), we will work with the extended population resulted from the execution of {\itshape crossover}(). We will firstly determine the determine the "fitness" of each chromosome/member of the population (a real positive number yielded by the expression $1/(0.00001 + Euclidian\_2D(decoded_chromosome)))$. Then we will apply elitism on the population (we will ensure that a few best chromosomes appear in the future generation at least once) and finally, with the use of a "roulette wheel", we will select the remaining $n-k$ chromosomes, where $n$ is the initial or standard size of the population and $k$ is the number of chromosomes selected with elitism.

After executing the three operations above for a certain number of generations, we will use the optimal tour found throughout all generations as starting point for the Hill Climbing Best Improvement, which will be used to further improve the result yielded by the algorithm.

For each map, we will run the algorithm $30$ times, with the following parameters: a population size of $100$, $1500$ allowed generation, a hypermutation threshold of $350$, a mutation rate of $0.015$, a crossover rate of $0.35$ and an elitism rate of $0.1$.

\section{Results}

\subsection{Tables}

\begin{table}[H]
\begin{adjustwidth}{}{}
\centering
\resizebox{\textwidth}{!}{
\begin{tabular}{*{7}{|l}|} \hline
\multicolumn{7}{|c|}{Results}\\\hline
Instance & \Centerstack[c]{Optimal\\ Value} & \Centerstack[c]{Lowest Value\\ Obtained} & \Centerstack[c]{Highest Value\\ Obtained} & Average & \Centerstack[c]{$\sigma$\\(standard deviation)} & Average Time \\\Xhline{3\arrayrulewidth}
eil51 & 426 & 452.39 & 504.52 & 484.1536 & 13.24391 &2.921166 s\\\hline
st70 & 675 & 857.882 & 949.251 & 908.1673 & 26.68194 & 8.730105 s\\\hline
krod100 & 21294 & 41337.4 & 52108.6 & 47588.5 & 2501.634 & 4.066486 s\\\hline
ts225 & 126643 & 543578 & 655030 & 621550.5 & 25043.06 & 14.2773 s\\\hline
\end{tabular}}
\caption{Results for Simulated Annealing}
\end{adjustwidth}
\end{table}

\begin{table}[H]
\begin{adjustwidth}{}{}
\centering
\resizebox{\textwidth}{!}{
\begin{tabular}{*{7}{|l}|} \hline
\multicolumn{7}{|c|}{Results}\\\hline
Instance & \Centerstack[c]{Optimal\\ Value} & \Centerstack[c]{Lowest Value\\ Obtained} & \Centerstack[c]{Highest Value\\ Obtained} & Average & \Centerstack[c]{$\sigma$\\(standard deviation)} & Average Time \\\Xhline{3\arrayrulewidth}
eil51 & 426 & 511.508 & 675.088 & 590.3399 & 43.02564 & 2.592588 s\\\hline
st70 & 675 & 970.972 & 1251.44 & 1097.229 & 66.90697 & 3.920526 s\\\hline
krod100 & 21294 & 35051.6 & 50003.3 & 41699.61 & 3799.186 & 7.01967 s\\\hline
ts225 & 126643 & 273575 & 356486 & 319754 & 23678.86 & 45.09334 s\\\hline
\end{tabular}}
\caption{Results for Genetic Algorithm}
\end{adjustwidth}
\end{table}

\subsection{Interpretation}

As we can see in the tables above, when it comes to instances with a number of nodes lower than 100, the Simulated Annealing yields slightly better results than the Genetic Algorithm, in the case of \textit{eil51} coming extremely close to the optimal value. On the hand hand, for instances with 100 or more nodes, the Genetic Algorithm reaches better values, while the Simulated Annealing cannot reach an attraction pool with a low value for the local minima, since the number of permutation is extremely large, and the greedy strategy it follows is unlikely to find close to optimal results. However, the results yielded by the Genetic Algorithm in this case are rather adequate than excellent.

As for execution time, for instances with less than 100 nodes, the Genetic Algorithm performs faster, while for instances with 100 or more nodes, it is the Simulated Annealing that computes the result quicker, in the case of \textit{ts225} the difference between the 2 methods being more than 30 seconds in average.

\section{Conclusions}

All things considered, we can state that Genetic Algorithm produces good results for an \textbf{NP}-hard problem and, due to the short execution time, it is usable in practice. Moreover, because GA is more complex than most of the heuristics, it has a higher number of parameters which can be modified, this giving us the possibility to change their value until we find the most suitable ones. 

\begin{thebibliography}{9}

\bibitem{tsp}
	Traveling Salesman Problem - From Wikipedia, the free encyclopedia $-$
	\url{https://en.wikipedia.org/wiki/Travelling_salesman_problem}

\bibitem{heuristic}
	Algoritmi Genetici - Metode traiectorie $-$
	\url{https://profs.info.uaic.ro/~pmihaela/GA/laborator2.html}

\bibitem{mutation}
	Suvarna Patil, Manisha Bhende, {\itshape Comparison and Analysis of Different Mutation Strategies to improve the Performance of Genetic Algorithm} $-$
	\url{https://pdfs.semanticscholar.org/75eb/30781a18a63a89b09af6cea9c388accac0d4.pdf}

\bibitem{instances}
	MP-TESTDATA - The TSPLIB Symmetric Traveling Salesman Problem Instances $-$
	\url{http://elib.zib.de/pub/mp-testdata/tsp/tsplib/tsp/index.html}


	\end{thebibliography}
\end{document}